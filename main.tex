\documentclass{abstract}
\begin{document}
\makeabstracttitle[
presentno={B9-99},
title={表題表題表題表題表題表題表題表題表\\題表題表題表題表題表題表題表題表題表題表題表題表題表題表題表題表題},
labname={サンプル研究室},
lastname={鳥取},
firstname={太郎},
]
%%%%%%%%%%%%%%%%%%%%%%%%%%%%%%%%%%%%%%%
\section{はじめに}
%%%%%%%%%%%%%%%%%%%%%%%%%%%%%%%%%%%%%%%
\section{関連研究}
参考文献の参照例を示す\cite{Article:2024:Toridai}.
%%%%%%%%%%%%%%%%%%%%%%%%%%%%%%%%%%%%%%%
\section{図}
\figref{fig:example-image}に例を示す.

\begin{figure}[t]
  \centering
  \includegraphics[width=\columnwidth]{example-image}
  \caption{図の例}
  \label{fig:example-image}
\end{figure}
%%%%%%%%%%%%%%%%%%%%%%%%%%%%%%%%%%%%%%%
\section{表}
\tabref{tab:example-example}に例を示す.

\begin{table}[t]
  \centering
  \caption{表の例}
  \begin{tabular}{|c c c|}
    a & b & c \\
    \hline
    00 & 01 & 02 \\ 
    10 & 11 & 12 \\ 
    20 & 21 & 22 \\ 
  \end{tabular}
  \label{tab:example-example}
\end{table}
%%%%%%%%%%%%%%%%%%%%%%%%%%%%%%%%%%%%%%%
\section{おわりに}
%%%%%%%%%%%%%%%%%%%%%%%%%%%%%%%%%%%%%%%
\bibliographystyle{junsrt}
\bibliography{main}
\end{document}
